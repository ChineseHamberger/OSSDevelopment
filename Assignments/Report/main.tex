\documentclass[lang=cn,11pt,a4paper,cite=authoryear]{elegantpaper}

\usepackage{tcolorbox}

\title{《开源软件技术》课程项目期末报告}
\author{
<姓名>\\<学号>\\\email{xxxxx@pku.edu.cn}
}
\date{}

\begin{document}

\maketitle

\section{开源项目概览}

\begin{tcolorbox}[title=\textbf{请在这一节包含如下内容:},colback=yellow!10!white]
  列出你做出贡献的开源项目及其链接;对每一个开源项目:
  \begin{enumerate}
    \item 请介绍其基本情况(例如,这个项目用于什么目的?是如何产生和发展的?成熟度和活跃度如何?是否存在特定的组织模式和商业模式?等等);
    \item 请介绍你决定向该项目做贡献的历程(例如,你是如何得知这个开源项目的?这个项目的什么地方吸引到了你?为什么对这个开源项目产生了贡献的动机?等等)。
  \end{enumerate}
\end{tcolorbox}

\section{对开源项目的贡献}

\begin{tcolorbox}[title=\textbf{请在这一节包含如下内容:},colback=yellow!10!white]
用一段话总结你提交的所有贡献,完成表格~\ref{tab:contrib};对每一个贡献,请在对应的小节里:
\begin{enumerate}
    \item 描述这是一个什么贡献;
    \item 描述这个贡献的定位和解决过程(如何定位确定要做这个贡献?与其他开发者做的沟通?遇到了什么困难?有没有经过多轮修改?等等);
    \item 在整个过程中,你的贡献经过了哪些质量检查(如果有)?(例如,添加新的单元测试、由核心开发者进行的代码审查,等等)
    \item 这个贡献是否被接受?如果没有,是为什么?
\end{enumerate}
\end{tcolorbox}

\begin{table}[]
    \centering
    \caption{对开源项目的贡献概览(已有内容仅供格式举例,请替换成你的贡献)}
    \begin{tabular}{llp{9cm}}
      \toprule
        项目 & 贡献 & 相关链接 \\
      \midrule
        GFI-Bot & 为\texttt{gfibot.data}模块添加日志 & \url{https://github.com/osslab-pku/gfi-bot/issues/13} \newline \url{https://github.com/osslab-pku/gfi-bot/pull/15}\\ 
      \bottomrule
    \end{tabular}
    \label{tab:contrib}
\end{table}

\subsection{贡献一:XXXXX}

\subsection{贡献二:YYYYY}

\subsection{...}

\section{总结与反思}

\subsection{计划外的变化}

\begin{tcolorbox}[title=\textbf{请在这一节包含如下内容(如果有):},colback=yellow!10!white]
\begin{enumerate}
    \item 你是否尝试对一些开源项目做贡献,但是最后失败了?为什么会失败?后来你又是如何切换开源项目的?
    \item 你是否尝试去解决一些开源项目的Issue,但是最后失败了?为什么会失败?
    \item 在开源项目的贡献过程中有哪些你没预期到的困难?你是如何克服这些困难的?
\end{enumerate}
\end{tcolorbox}

\subsection{对开源的体会、理解与思考}

\begin{tcolorbox}[title=\textbf{请在这一节中写下任何你想要表达的内容,下述问题仅供参考:},colback=yellow!10!white]
在完成课程项目的过程中,你对开源模式和开源项目产生了哪些印象和体会?从参与开源中,你学到了什么东西?未来是否会继续参与开源?
\end{tcolorbox}

\subsection{对课程的建议}

\begin{tcolorbox}[colback=yellow!10!white]
请在这里写下任何对课程的意见和建议(可选,不计入报告评分)。
\end{tcolorbox}


\nocite{*}
\bibliography{references}

\end{document}
